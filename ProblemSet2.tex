%\documentclass[11pt,reqno]{amsart}
\documentclass[11pt,reqno]{article}
\usepackage[margin=.8in, paperwidth=8.5in, paperheight=11in]{geometry}
%\usepackage{geometry}                % See geometry.pdf to learn the layout options. There are lots.
%\geometry{letterpaper}                   % ... or a4paper or a5paper or ... 
%\geometry{landscape}                % Activate for for rotated page geometry
%\usepackage[parfill]{parskip}    % Activate to begin paragraphs with an empty line rather than an indent7
\usepackage{graphicx}
\usepackage{pstricks}
\usepackage{amssymb}
\usepackage{epstopdf}
\usepackage{amsmath}
\usepackage{subfigure}
\usepackage{caption}
\pagestyle{plain}
%\renewcommand{\topfraction}{0.3}
%\renewcommand{\bottomfraction}{0.8}
%\renewcommand{\textfraction}{0.07}
\DeclareGraphicsRule{.tif}{png}{.png}{`convert #1 `dirname #1`/`basename #1 .tif`.png}

\title{Foundations of Computer Architecture: \\ Problem Set 2 }
\author{Andrew Rickert}
\date{Professor: Dr. Horace Malcolm \\ \hspace{-19pt} Due Date: February 14,  2012}                                           % Activate to display a given date or no date

\begin{document}
\maketitle


% Page 1
\begin{flushleft} 
Problem 1 \\
\rule{500pt}{1pt}\\
\end{flushleft} 
\framebox{Part a}\\ 
 First we convert the decimal values to their two's complement representation:
 \begin{eqnarray*}
-2147483648_{ten} &=& 10000000000000000000000000000000_{two} \\
-4_{ten} &=& 11111111111111111111111111111100_{two} \\
\end{eqnarray*}

\noindent The instruction \emph{addu} \$8,\$9,\$8 will add the contents of \$9 to the contents of \$8 and store the result back in \$8. This is an unsigned addition so there will be no overflow exception. Summing the values in the registers gives the following result
\[
\$8 = 01111111111111111111111111111100_{two} = 2147483644_{ten}\\
\]

\noindent\framebox{Part b}\\ 
\noindent The instruction \emph{xor} \$9,\$8,\$9 will perform an exclusive-or on the contents of \$8 with the contents of \$9 and store the result back in \$9. Performing the operation gives the following result.
\[
\$9 = 01111111111111111111111111111100_{two} = 2147483644_{ten}\\
\]
In effect xoring two registers results in the same value as an addition without an overflow exception.
\begin{flushleft} 
Problem 2 \\
\rule{500pt}{1pt}\\
\end{flushleft} 
\framebox{Part a}\\ 
In an excess-$K$ system the intended value is $K$ less than the unsigned binary value. As an unsigned binary value we have
\[ \text{0xAFFFFFFF}_{hex} = 10101111111111111111111111111111_{two} = 2952790015_{ten} \]
So the represented value must be $2952790015-2147483648 = 805306367_{excess-2147483648}$\\

\noindent \framebox{Part b}\\ 
Since the intended value in an excess-$K$ system is $K$ less than the unsigned binary value and the smallest unsigned binary value is 0 then the small representable value is $-65540$.\\

\noindent \framebox{Part c}\\ 
We indicate below for which given system -128 maybe represented.\\

\noindent I. \hspace{12pt}Excess-128 system\\
\indent Yes, -128 is represented as 00000000.\\

\noindent II. \hspace{7pt}One's complement\\
\indent No, the smallest value is -127.\\

\noindent III. \hspace{3pt}Sign and Magnitude\\
\indent No, the smallest value is -127\\

\noindent IV. \hspace{3pt}Two's complement\\
\indent Yes, -128 is represented by 11111111\\

\begin{flushleft} 
Problem 3 \\
\rule{500pt}{1pt}\\
\end{flushleft} 
First we need to convert the decimal value to binary. We keep the leading zeros in the binary expression below to make the number 64 bits.
\[ 3904679375210100017_{ten} = 0011011000110000001101010011010000110001001100010011010100110001_{two}\]

\noindent Now we need to convert each set of 8 bits to its corresponding decimal then match it to the table that defines the decimal to ASCII conversion.\\

\begin{tabular}{| l | c | c | l |}
\hline
Byte Address & Binary Value & Decimal Value &ASCII character represented by contents \\ \hline
0x10080000 & 00110110 & 54 & 6\\ \hline
0x10080001 & 00110000 & 48 & 0\\ \hline
0x10080002 & 00110101 & 53 & 5\\ \hline
0x10080003 & 00110100 & 52 & 4\\ \hline
0x10080004 & 00110001 & 49 & 1\\ \hline
0x10080005 & 00110001 & 49 & 1\\ \hline
0x10080006 & 00110101 & 53 & 5\\ \hline
0x10080007 & 00110001 & 49 & 1\\ \hline
\end{tabular}

\begin{flushleft} 
Problem 4 \\
\rule{500pt}{1pt}\\
\end{flushleft} 
The table below indicates whether a given item is an architectural feature or an organizational feature.\\

\begin{tabular}{| l | c | c |}
\hline
Item & Architectural feature & Organizational feature \\ \hline
Cache memory & & \checkmark \\ \hline
Multiplexed I/O bus & & \checkmark \\ \hline
Indexed addressing mode & \checkmark & \\ \hline
128-bit CPU registers & \checkmark & \\ \hline
Flash based storage & & \checkmark \\ \hline
Vector instruction & \checkmark & \\ \hline
Micro-programmed control & & \checkmark \\ \hline
USB-serial I/O port & & \checkmark \\ \hline
\end{tabular}

\begin{flushleft} 
Problem 5 \\
\rule{500pt}{1pt}\\
\end{flushleft} 
\framebox{Part a}\\ 

\noindent \framebox{Part b}\\ 



\noindent \framebox{Part c}\\ 



\noindent \framebox{Part d}\\ 



\noindent \framebox{Part e}\\ 



\end{document}  